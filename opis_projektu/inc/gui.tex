\section{Interfejs graficzny}
W celu stworzenia GUI wykorzystano biblioteki \textit{Qt}. Interfejs graficzny został zaprojektowany z wykorzystaniem programu \textit{QtDesigner}.

\subsection{Wykorzystane klasy}
Tworząc interfejs wykorzystano biblioteki:

\begin{itemize}
\item \textbf{QGraphicsScene} - biblioteka pozwalająca na stworzenie obiektu reprezentującego dwuwymiarową scenę graficzną, na scenie można tworzyć i umieszczać obiekty (pojedyncze piksele, wieloboki), obiekty można poddawać transformacji przy pomocy klasy \textit{QTransform},
\item \textbf{QGraphicsView} - biblioteka, przy pomocy której można wizualizować zawartość sceny utworzonej z wykorzystaniem klasy \textit{QGraphicsScene}.
 \end{itemize}

\subsection{Proponowany interfejs}
\begin{figure}[H]
  \centering
  \includegraphics[width=0.5\textwidth]{img/okno.png}
\end{figure}
