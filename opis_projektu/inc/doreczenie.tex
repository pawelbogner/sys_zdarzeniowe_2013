\label{sec:doreczenie}
Plan dzia�ania z wyr�nionymi elementami oddawanych raport�w:
\begin{enumerate}
	\item Monta� elektroniki: 
	\begin{itemize}
 		\item schemat po��czenia element�w,
		\item schemat p�ytki PCB,
		\item rozmieszczenie element�w na p�ytce,
		\item zdj�cia wykonanego modelu,
		\item projekt obudowy
	\end{itemize}
	
	{\linia}
	\begin{flushright}
	  termin: 12.03.2012
	 \end{flushright}
	 
 	\item Oprogramowanie elektroniki:
	\begin{itemize}
 		\item kod �r�d�owy programu steruj�cego elektronik�,
	\end{itemize}

 	\item Matematyka filtru Kalmana:
	\begin{itemize}
		\item model matematyczny filtru Kalmana,
		\item projekt algorytmu  do prze�o�enia w MATLABie/LabVIEW,
	\end{itemize}

	{\linia}
	\begin{flushright}
	  termin: 27.03.2012
	 \end{flushright}

 	\item MATLAB:
	\begin{itemize}
		\item skrypt do akwizycji w MATLABie,
		\item skrypt do filtracji w MATLABie,
		\item skrypt do wizualizacji w MATLABie,
	\end{itemize}
 
	\item LabVIEW:
	\begin{itemize}
 		\item skrypt do akwizycji w LabVIEW,
		\item skrypt do filtracji w LabVIEW,
 		\item skrypt do wizualizacji w LabVIEW,
	\end{itemize}

	{\linia}
	\begin{flushright}
	  termin: 24.04.2012
	 \end{flushright}

 	\item Dokumentacja:
	\begin{itemize}
 		\item strona internetowa zawieraj�ca dokumentacje projektu,
 		\item sprawozdanie zredagowane w {\LaTeX} zawieraj�ce
 		opisy elementy projektu, bez kod�w �r�d�owych
	\end{itemize}

	{\linia}
	\begin{flushright}
	  termin: 8.05.2012
	 \end{flushright}
	 
\end{enumerate}

Wszystkie elementy raport�w maj� charakter jawny. 
% chyba nie ustalili�my w ko�cu jak mamy nazwa� stopie� jawno�ci

Poszczeg�lne raporty b�d� oddawane na najbli�szych 
spotkaniach projektowych, po sko�czeniu opracowywania 
zagadnie� w terminach zgodnych z wykresem Gantta i terminami podanymi powy�ej.