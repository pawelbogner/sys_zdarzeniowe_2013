\noindent Zdecydowano, że zostanie wykorzystana tradycyjna, płaska struktura
zarządzania z jednym liderem (koordynatorem).  Do zadań lidera należy
podejmowanie krytycznych decyzji projektowych, rozstrzyganie sporów
oraz kontrolowanie postępu prac nad przydzielonymi zadaniami.
\\\\
\noindent W kwestii rozstrzygania sporów, strona konfliktu ma prawo do
przedstawienia problemu na forum grupy, w celu jego wspólnego
przedyskutowania.  W świetle przedstawionych argumentów i poglądów
lider ma obowiązek podjąć decyzję rozstrzygającą.
\\\\
\noindent Za termin regularnych spotkań przyjęto termin zajęć odbywających się w
ramach kursu ,,Systemy zdarzeniowe''. Lider zespołu ma prawo
zarządzenia dodatkowego spotkania organizacyjnego na wniosek jednego
lub kilku członków zespołu.  Osoba wnosząca o zorganizowanie zebrania
ma obowiązek wykazać, że spotkanie jest niezbędne w celu dalszego
rozwoju projektu.
\\\\
\noindent Do przewidzianych środków komunikacji zdalnej należą ,,Google
groups'', ,,Redmine'' oraz rozmowy telefoniczne.  W celu składowania i
wymiany dokumentów zostanie wykorzystane oprogramowanie ,,SVN''.
Każdy z członków grupy na obowiązek korzystać z tego programu.  Jako
mechanizm monitorowania postępów prac zostanie wykorzystana usługa
,,Redmine''.  Jest to narzędzie, które pozwala przydzielić zadanie
danemu członkowi lub członkom zespołu, określić ramy czasowe jego
wykonania, oraz śledzić postęp prac.
\\\\
\noindent Uznano, że każdy członek grupy zostanie obdarzony prawami własności
intelektualnej do części projektu, za której zrealizowanie był
odpowiedzialny.
