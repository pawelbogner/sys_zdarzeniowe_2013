\section{Zarządzanie projektem}
\noindent Podczas realizacji projektu wykorzystano tradycyjną, płaską strukturę
zarządzania z jednym liderem (koordynatorem). Do zadań lidera należało
podejmowanie krytycznych decyzji projektowych, rozstrzyganie sporów
oraz kontrolowanie postępu prac nad przydzielonymi zadaniami.
\\\\
\noindent W kwestii rozstrzygania sporów, strona konfliktu ma prawo do
przedstawienia problemu na forum grupy, w celu jego wspólnego
przedyskutowania. W świetle przedstawionych argumentów i poglądów
lider ma obowiązek podjąć decyzję rozstrzygającą.
\\\\
\noindent Do przewidzianych środków komunikacji zdalnej należą \textit{Google
groups} oraz rozmowy telefoniczne. W celu składowania i
wymiany dokumentów zostanie wykorzystane oprogramowanie \textit{git}.
Każdy z członków grupy ma obowiązek korzystać z tego programu.
\\\\
\noindent Uznano, że każdy członek grupy zostanie obdarzony prawami własności
intelektualnej do części projektu, za której zrealizowanie był odpowiedzialny.
