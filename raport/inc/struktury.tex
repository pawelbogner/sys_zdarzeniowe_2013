
\section{Zarys struktur danych}

\subsection{Scena}

W zrealizowanym modelu scena (plansza) jest strukturą, zawierającą tablicę komórek. Składa się ona na wszystkie 
niezbędne dane do sterowania robotami.\\

Pojedyncza komórka zawiera informacje o swoich wymiarach oraz listę wskaźników na roboty, które aktualnie się w niej znajdują, a także interfejs do dodawania i usuwania robotów w zależności od pojawiających się zdarzeń (roboty wjeżdżają i wyjeżdżają z komórki).\\

Klasa opisująca pojedynczego robota zawiera informacje o położeniu robota, jego prędkościach, a także niezbędne metody do wyliczania sił sterujących robotem (zależnych od położenia robota względem środka sektora i położenia ewentualnego drugiego robota w sektorze).

\subsection{Dane wymieniane z serwerem}

Serwer ma zadanie planowania tras dla robotów, wobec czego pożądana informacja dla każdego z robotów to następna komórka, do której ten ma się kierować oraz zezwolenie bądź brak zezwolenia na wjazd do niej. W przypadku braku zezwolenia robot zatrzymuje się w bieżącej komórce; w przeciwnym wypadku przejeżdża od razu do następnej komórki.\\

Klient wysyła zapytania o dalsze drogi robotów oraz o pozwolenie na ich realizację, a także informuje o wykonanych zadaniach oraz o zwalnianych komórkach.
